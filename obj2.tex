\chapter{GP-based Approaches for On-line \emph{RAC}}


\section{Dataset}
\begin{lstlisting}
-- PH.D_project/data/baseConfig            # PM and VM configurations
-- PH.D_project/data/containerData/dynamic/  # training and test instances
-- PH.D_project/data/InitEnv/              # Initial data center
-- PH.D_project/data/OSData/dynamic/	   # The OS requirements for the containers in the test instances 
-- PH.D_project/data/OSPro/			# OS probabilities
\end{lstlisting}

\begin{lstlisting}
InitEnv
	-- container.csv   # The initial containers
	-- os.csv		   # The OS of VMs	
	-- pm.csv		   # The initial PMs
	-- vm.csv          # The initial VMs
\end{lstlisting}

\begin{lstlisting}
container.csv

#CPU, #Mem
26,98.3
175.56,425.98
27.73,252.58
6.93,125.61
\end{lstlisting}

\begin{lstlisting}
os.csv

# OS type of each VM
1
2
2
...
\end{lstlisting}

\begin{lstlisting}
pm.csv

# Each row represents a PM. The numbers in each row represent the type of VM in this PM

10,10
6,1
9,2,4,9
2
2
...
\end{lstlisting}

\begin{lstlisting}
vm.csv

# Each row represents a VM. The numbers in each row represent the indexes of containers.

1,2,3,4
5,6
7,8,9,10
11
12,13,14,15
...
\end{lstlisting}

\begin{lstlisting}
-- PH.D_project/data/OSPro/OS5/probability.csv

# the probability of each type of OS
0.179
0.454
0.236
0.105
0.026
\end{lstlisting}


\section{Algorithm Code and Scripts}
\subsection{Algorithm Code}

\begin{lstlisting}
Directories
-- ContainerAllocationCCGP # For CCGP training
-- ContainerAllocationGPHH # For GPHH training
-- containerAllocation 	   # For simulation, or testing
\end{lstlisting}


\subsection{Scripts}
\subsubsection{For Test Instance Generation}
\begin{lstlisting}
Directories:
-- PHD_project/dataGenerationScripts/generateTaskCases.R          
# For generating individual test instances

-- PHD_project/data/dataGenerationScripts/generateOSCases.R       
# For generating the OS requirements for these test instances

-- PHD_project/data/dataGenerationScripts/generateBaseDataCenter.R  
# For generating the initial data center
\end{lstlisting}


\begin{flushleft}\textbf{generateTaskCases.R}\end{flushleft}

\begin{lstlisting}
Usages:

generateTask(
	whichDataSet=[1:auvergrid | 2:bitbrains], 
	whichVMsize=[ 6:ten types | 7:twenty types ], 
	size = 2500, 
	testCase, filename='')

Example: 

for (i in seq(0,199)) 
	generateTask(1, 6, 2500, i, paste('./container2500/testCase', i, '.csv', sep=''))")
\end{lstlisting}


\begin{flushleft}\textbf{generateOSCases.R}\end{flushleft}

\begin{lstlisting}
generateOScases(
	size=2500, 
	numOfOS=[3 | 4 | 5], 
	testCase, 
	filename='')

Example:

for(i in seq(0, 199)) 
	generateOScases(2500, 5, i, paste('./OS1Container1000/testCase', i, '.csv', sep=''))")
\end{lstlisting}

\begin{flushleft}\textbf{generateBaseDataCenter.R}\end{flushleft}

\begin{lstlisting}
generateBaseDataCenter(
	testCaseSize=2500, 
	OSNum=[3 | 4 | 5], 
	testCase)

Example:

for(i in seq(0, 199)) 
	generateOScases(2500, 5, i)
\end{lstlisting}


\subsubsection{For Run Experiments}
The grid scripts for GPHH and CCGP training,
\begin{lstlisting}
Directories:
-- PHD_project/gridScripts/CCGP/ccgp.sh   # The configuration scripts
-- PHD_project/gridScripts/CCGP/run.sh	  # submit 40 runs

-- PHD_project/gridScripts/GPHH/gphh.sh
-- PHD_project/gridScripts/GPHH/run.sh
\end{lstlisting}


\begin{lstlisting}
Usage:

./run.sh [vmtypes = TEN | TWENTY] [OSNUM = 5] [dataset = BITBRAINS | AUVERGRID]

Example:

./run.sh TEN 5 BITBRAINS
\end{lstlisting}

\begin{flushleft}\textbf{The scripts for data collection:}\end{flushleft}

\begin{lstlisting}
-- dynamicContainerAllocation/scriptCCGP/execute.sh
--	...
-- dynamicContainerAllocation/scriptGPHH/execute.sh
--  ...
\end{lstlisting}


\begin{lstlisting}
Usage:

./collect.sh [alg = CCGP] [osNum=5] [vmNum=TWENTY | TEN] [dataset=AUVERGRID | BITBRAINS]

Example:

./collect.sh CCGP 5 TWENTY AUVERGRID
\end{lstlisting}



\begin{flushleft}\textbf{The scripts for testing:}\end{flushleft}
Before test, the simulation must be correctly configured in the program.


\begin{flushleft}The example shows how to test a GPHH rule on the Auvergrid dataset with 10 VM types.\end{flushleft}
\begin{lstlisting}
In containerAllocation/src/Experiment_selection_creation_combined.java:

        experimentRunner(
                run,
                folder,
                TestDataSet.AUVERGRID_TEN,
                TestCaseSizes.xLarge,
                OperatingSystems.THREE,
                FitnessFunctions.EVO, // vm selection rule
                FitnessFunctions.SUB, // pm selection rule
                VMCreationRules.EVO,
                VMAllocationFramework.ANYFIT, 	// vm selection-creation framework, when 
                				// select NONANYFIT framework,
                                                // the VMCreationRules will be 
                                                // automatically canceled.
                SelectionRules.FIRSTFIT, 	// pm selection framework,
                                        	// If we choose FirstFit, then, the fitness functions 
                                        	// of pm selection rule will be ignored
                PMCreationRules.LARGEST); 	// No other choice
    }
\end{lstlisting}

\begin{flushleft}After compiling the code, you may use the testing script to test.
Then, the testing script:\end{flushleft}

\begin{lstlisting}
--PH.D_project/experimentScripts/dynamicAllocation/experimentRunningScripts/runSingleExperiment.sh
--PH.D_project/experimentScripts/dynamicAllocation/experimentRunningScripts/test30Runs.sh
\end{lstlisting}


\begin{lstlisting}
Usage:

./test30Runs.sh [Path]

Example:

./test30Runs.sh ./Container2500_AUVERGRID_TEN  
# where the collected rules are in the Container2500_AUVERGRID_TEN
\end{lstlisting}


\subsubsection{For Data Analysis}
\begin{lstlisting}
--PH.D_project/experimentScripts/dynamicAllocation/statScripts/
	-- barPlotEvoCompare.R  # plot bar chart to compare GPHH and CCGP
	-- barPlotPm.R          # plot bar chart of PM number
	-- barplotVmDist.R      # plot bar chart of VM distribution
	-- calVmNum.R           # calculate the number of VMs used
	-- energyAveSd.R        # calculate the mean and sd of a scenario
	-- meanPmRemain.R 		# calculate the mean PM resource remaining
	-- meanPmUtil.R   		# calculate the mean PM resource utilization
	-- plotWaste.R          # plot line chart of waste resources
	-- stat_of_30_general.R # statistic significant test
	-- vmDist20.R           # calculate the frequency of 20 VM types
	-- vmDist.R             # calculate the frequency of 10 VM types
\end{lstlisting}

The usage of each script is record in the scripts and will be printed out when first load them.
