\chapter{EC for Multi-objective \emph{RAC}}

\section{Dataset}
\begin{lstlisting}
Dataset Directories:
-- data/containerData/static/multi_objective
-- data/containerData/static/spaceEstimation
\end{lstlisting}

% \begin{lstlisting}
% PM Configuration Files:
% -- data/baseConfig/PMConfig/PMConfig_small.csv
% \end{lstlisting}

% \begin{lstlisting}
% VM Configuration Files:
% -- baseConfig/VMConfig/LVMnePM/VMConfig_ten.csv
% -- baseConfig/VMConfig/LVMnePM/VMConfig_twenty.csv
% \end{lstlisting}


% \subsection{Configuration Files}


% \begin{lstlisting}
% Algorithm Configuration Files:
% -- baseConfig/VMConfig/LVMnePM/VMConfig_ten.csv
% -- baseConfig/VMConfig/LVMnePM/VMConfig_twenty.csv
% \end{lstlisting}

% \subsection{Application Files}
The application files describes the information of applications including the number of applications,
each application is consist of how many micro-services, and the number of replicas for each micro-services. 
The application files are named as \textit{application} + \textit{number of application} + .csv.
\begin{lstlisting}
application file
			
application50.csv
application100.csv
application150.csv
application200.csv
\end{lstlisting}



The data structure of application files are :
\begin{lstlisting}
micro-service 1, micro-service 2, ..., micro-service 5
2,4,0,0,0
3,0,0,0,0
3,4,4,2,5
3,4,2,3,2
\end{lstlisting}

Each row represents an application. 
Each column represents a micro-service. In the current setting, each
application has maximum of 5 micro-services. The number of each index denotes the number of replicas of that micro-service. For example, in the first row of the above dataset, [2, 4, 0, 0, 0] 
denotes application 1 is consist of 2 micro-services. Micro-service 1 has 2 replicas and micro-service 2 has 4 replicas.

The constraint for the number of replicas is defined as following. 
The lower bound is 2. 
The upper bound is 5.
0 means the micro-service does not exist.

 
\subsubsection{TestCase Files}

The testCase files contain the detailed resource requirement of replicas.
The testCases and application files are 1-on-1 mapping. 
For example, \textit{testCase50.csv} and \textit{application50.csv} are describing the same instance. 
The application files are named as \textit{application} + \textit{number of application} + .csv.
 meaning application50.csv
\begin{lstlisting}
# testCase file 

testCase50.csv
testCase100.csv
testCase150.csv
testCase200.csv
\end{lstlisting}



\begin{lstlisting}
# testCase data structure

CPU, MEM, applicationId, micro-serviceId, replicaId
53.73,244.39,1,1,1
53.73,244.39,1,1,2
255.54,410.96,1,2,1
255.54,410.96,1,2,2
\end{lstlisting}

Notice that, in experiments of variable size of replicas, the number of
replicas is determined by algorithm. Although we are using the same dataset,
the number of replica is not used. 


\section{Algorithm Code and Scripts}

\subsection{Algorithm Code}
\begin{lstlisting}
-- GA/src/LNS_FF        	# Large-neighbourhood search and FF
-- GA/src/LNS_NS-GGA    	# LNS and NS-GGA
-- GA/src/NSGAII_NS-GGA 	# NSGAII and NS-GGA
-- GA/src/NSGAII_Spread 	# NSGAII and Spread
-- LNBPSO/src/LNS_BPSO		# LNS and BPSO
\end{lstlisting}
\subsection{Scripts}

\subsubsection{Grid}
\begin{lstlisting}
-- PH.D_project/gridScripts/multiObj_VariableReplicas/run_ga.sh
-- PH.D_project/gridScripts/multiObj_VariableReplicas/ga.sh
-- PH.D_project/gridScripts/multiObj_VariableReplicas/run_bpso.sh
-- PH.D_project/gridScripts/multiObj_VariableReplicas/bpso.sh
\end{lstlisting}

\begin{lstlisting}
run_ga.sh

Usage:

./run_ga.sh [algorithm = NSGAII_NSGGA | LNS_FF | NSGAII_Spread | LNS_NSGGA ] 
[appSize = app100 | app150 | app200] [vmTypes = ten | twenty]
\end{lstlisting}

\begin{lstlisting}
run_bpso.sh

Usage: 

./run.sh [algorithm = NSGAII_NSGGA | LNS_FF | NSGAII_Spread | LNS_NSGGA ] 
[appSize = app100 | app150 | app200] [vmTypes = ten | twenty]

\end{lstlisting}

\subsubsection{For Data Collection}

\begin{lstlisting}
-- PH.D_project/experimentScripts/multiObjectiveStaticAllocation/collect.sh			
# This is the main collection script. It will execute all the sub-scripts
-- ...sh
\end{lstlisting}

\begin{lstlisting}
Usage 

./collect.sh [alg = multiGroupGA] [vmTypes = ten | twenty] [appNum = 150]
Example:

./collect.sh multiGroupGA ten 150
\end{lstlisting}

\subsubsection{For Data Analysis}


